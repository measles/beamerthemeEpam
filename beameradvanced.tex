\input{common}
\title[bash]{Продвинутый пример использования темплейта для beamer-презентации}
\subtitle{Как управлять отображением элементов на слайде и рисовать в \LaTeX}

\begin{document}

\begin{frame}
	\titlepage
\end{frame}

\begin{frame}{Table of contents}
	\tableofcontents
\end{frame}



%%%%%%%%%%%%%%%%%%%%%%%%%%%%%%%%%%%%%%%%%   
%%%%%%%%%% Content starts here %%%%%%%%%%
%%%%%%%%%%%%%%%%%%%%%%%%%%%%%%%%%%%%%%%%%

\section{Overlay}
\subsection{Pause}
\begin{frame}{Pause frame on specified point}
    This is the simplest way to control how single frame will be represented in slides: \lstinline[language=tex]{\\pause}\\~\\\pause

    This single frame will produce series of slides. Each \lstinline[language=tex]{\\pause} command will indicates the end of the slide.\\~\\\pause

    By the way, notice that indicated number for each slide of this series will have same value. This is not a slide number, actually, it's a frame number. 
\end{frame}

\subsection{Lists}
\begin{frame}
    \frametitle{List: appear one by one}
    On this frame we'll make list items appear on following slides. Behold!
    \begin{enumerate}
	\item<1-> This one will appear first
	\item<2-> Then goes this one
	\item<3-> Next this one and the last one
	\item<4-> Will appears with the next one
	\item<4-> Going along with previous one
	\item<3-> Last one, will appears on the same slide as a third one
    \end{enumerate}
\end{frame}

\begin{frame}
    \frametitle{Same for itemized list}
    And, of course, all that works fine with itemize list. Numbers after \lstinline[language=tex]{\\item} means slides of frame this particular list appears on.
    \begin{itemize}
	\item<1-> First
	\item<2-> Second
	\item<3-> Third
	\item<2-> Second
    \end{itemize}
\end{frame}

\begin{frame}
    \frametitle{Items: appear one specified slides}
    On this frame we'll make list items appear on specified slides only. 
    \begin{enumerate}
	\item<1,3> This one will appears on first and third slides
	\item<2,4-6> This one appears on the second and from forth till six one
	\item<2,5-> This one will appear on second slide and than from the fifth till last one
	\item<3-5> Will appears on slides from third till fifth one
	\item<4-> From forth one on
	\item<1,3,5> Will appears on slides one, three, five
    \end{enumerate}
\end{frame}

\begin{frame}
    \frametitle{Short notation}
    But what if we need list items to appear one by one? Then, of course, it is possible to set number manually, but it's easier to use short notations:
    \begin{columns}
    \column{0.4\textwidth}
	\begin{enumerate}
	    \item<+-> First
	    \item<+-> Second
	    \item<+-> Third
	    \item<+-> Fourth
        \end{enumerate}
    \column{0.4\textwidth}
        \begin{itemize}[<+->]
	    \item First
	    \item Second
	    \item Third
	    \item Fourth
        \end{itemize}
    \end{columns}

    ~\\~\\The lists will appear in the order they was described in frame code.
\end{frame}
\end{document}
